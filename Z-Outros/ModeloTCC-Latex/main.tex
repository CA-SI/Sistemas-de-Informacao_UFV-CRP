%%==============================================================
%%      Modelo de TCC para o curso de Sistemas de Informação
%%    Universidade Federal de Viçosa - Campus de Rio Paranaíba
%%
%%  Autor: Rodrigo Smarzaro (smarzaro@ufv.br)
%%         Centro Acadêmico de Sistemas de Informação
%%  Última versão Outubro de 2019
%%  Licença: MIT
%%
%%  Codificação UTF-8
%%  Requisitos:
%%              - pdftex
%%              - arquivo abntex2-UFV.sty
%%              - pygments (python pip)
%%==============================================================

% ---
% CONFIGURAÇÕES GLOBAIS
% ---

% Configurações Especiais
\documentclass[
	% -- opções da classe memoir --
	12pt, % tamanho da fonte
	openright, % capítulos começam em pág ímpar (insere página vazia caso preciso)
	oneside, % para impressão só no anverso. Oposto a twoside
	a4paper, % tamanho do papel.
  % -- opções do pacote abntex2 --
  % chapter=TITLE, % Títulos em maiúsculas
  sumario=tradicional, % Sumário padrão memoir (mais bonito "imo")
  % -- opções do pacote babel --
	english, % idioma adicional para hifenização
	brazil, % o último idioma é o principal do documento
]{abntex2} % Personaliza a capa. Precisa do arquivo ufv.cls para funcionar.
\newcommand{\comment}[1]{} % Adiciona o comando de comentários em multiplas linhas 

% Pacotes Fundamentais
\usepackage{misc/abntex2-UFV} % Personalização para a Universidade Federal de Viçosa
\usepackage{lmodern} % Usa a fonte Latin Modern			
\usepackage[T1]{fontenc} % Seleção de codigos de fonte de saída
\usepackage[utf8]{inputenc} % Codificação do documento (conversão automática dos acentos)
\usepackage{indentfirst} % Indenta o primeiro parágrafo de cada seção
\usepackage{graphicx} % Inclusão de gráficos
\usepackage{booktabs} % \toprule, \midrule e \bottomrule para tabelas
\usepackage[alf,abnt-emphasize=bf]{abntex2cite} % Sistema autor-data com títulos nas referências em negrito
% ---


% ---
% CONFIGURAÇÕES SOBRE O TRABALHO
% ---

% Dados para Capa, Folha de Rosto e Folha de Aprovação
\titulo{Template para Trabalhos de Conclusão de Curso da UFV}
\autor{Rodrigo Smarzaro e CASI}
\local{Rio Paranaíba}
\data{2019}
\orientador{Nome do Orientador}
%\coorientador{Nome do Coorientador}
\instituicao{Universidade Federal de Viçosa}
\campus{\emph{Campus} de Rio Paranaíba}
\curso{Sistemas de Informação}
\membrobancaA{Membro da Banca A}
\membrobancaB[UFMG]{Membro da Banca B}
\databanca{\today}

% Preambulo
% Atenção deve conter o tipo do trabalho, o objetivo, o nome da instituição e a área de concentração
\preambulo{Monografia apresentada à Universidade Federal de Viçosa como parte das exigências para a aprovação na disciplina Trabalho de Conclusão de Curso I}

% Configurações de aparência do PDF final
\makeatletter
% Metadados
\hypersetup{
		pdftitle={\@title},
		pdfauthor={\@author},
    pdfsubject={\imprimirpreambulo},
	  pdfcreator={LaTeX with abnTeX2},
		colorlinks=true, % false: links em frame; true: links coloridos
    linkcolor=black, % cor dos links no documento
    citecolor=blue, % cor dos links para a bibliografia
    filecolor=magenta, % cor dos links para arquivos
		urlcolor=blue, % cor dos links para sites
		bookmarksdepth=4 % profundidade do sumário do PDF
}
\makeatother
% ---

% ----------------------------------------------------------
% ELEMENTOS PRÉ-TEXTUAIS
% ----------------------------------------------------------
\begin{document}
\frenchspacing
\pretextual

% Capa
\imprimircapa

% Folha de Rosto
\imprimirfolhaderosto

% Inserir Folha de Aprovação
%\imprimirfolhadeaprovacao

% Dedicatória
\comment{ % Remova essa linha para descomentar
\begin{dedicatoria}
   \vspace*{\fill}
   \centering
   \noindent
   \textit{Texto qualquer da dedicatória} % Remova essa linha, substitua pela sua dedicatória
   \vspace*{\fill}
\end{dedicatoria}
} % Remova essa linha para descomentar

% Agradecimentos
\comment{ % Remova essa linha para descomentar
\begin{agradecimentos}
  \textit{Texto qualquer da dedicatória} % Remova essa linha, substitua pelo seu agradecimento
\end{agradecimentos}
} % Remova essa linha para descomentar

% Epígrafe
\comment{ % Remova essa linha para descomentar
\begin{epigrafe}
    \vspace*{\fill}
  	\begin{flushright}
		\textit{``Word? nunca mais.''\\ % Remova essa linha, substitua pela sua epígrafe
		(Qualquer usuário de \LaTeX)} % Remova essa linha, substitua pela sua epígrafe
	\end{flushright}
\end{epigrafe}
\end{agradecimentos}
} % Remova essa linha para descomentar

% ---
% Resumos
% ---

% Resumo em Português
\begin{resumo}
  \noindent
  \textit{Texto do seu resumo} % Remova essa linha, substitua pelo seu resumo
  \vspace{\onelineskip}
  \noindent
  \textbf{Palavras-chaves}: Trabalho de Conclusão de Curso, abntex, LaTeX, UFV.
\end{resumo}

% Resumo em Inglês
\begin{resumo}[Abstract]
  \begin{otherlanguage*}{english}
    \noindent
    \textit{Abstract text} % Remova essa linha, substitua pelo seu abstract
    \vspace{\onelineskip}
    \noindent
    \textbf{Key-words}: Term Paper, abntex, LaTeX, UFV.
  \end{otherlanguage*}
\end{resumo}
% ---


% Lista de Ilustrações
\pdfbookmark[0]{\listfigurename}{lof}
\listoffigures*
\cleardoublepage

% Lista de Tabelas
\pdfbookmark[0]{\listtablename}{lot}
\listoftables*
\cleardoublepage

% Lista de Siglas e Abreviaturas (opcional)
% Sintaxe: \item [sigla] Descrição da sigla
\comment{ % Remova essa linha para descomentar
\begin{siglas}
  \item[ABNT] Absurdas Normas Técnicas
  \item[UFV] Universidade Federal de Viçosa
  \item[CRP] \emph{Campus} de Rio Paranaíba
\end{siglas}
} % Remova essa linha para descomentar

% Lista de símbolos (opcional)
% sintaxe: \item [simbolo] Descrição do símbolo
\comment{ % Remova essa linha para descomentar
\begin{simbolos}
  \item[$\infty$] Infinito
\end{simbolos}
} % Remova essa linha para descomentar

% Sumario
\pdfbookmark[0]{\contentsname}{toc}
\tableofcontents*
\cleardoublepage
% ---


% ----------------------------------------------------------
% ELEMENTOS TEXTUAIS
% ----------------------------------------------------------

\textual

% Modifique a estrutura dos capítulos e seções de acordo com a necessidade do seu trabalho
\chapter{Introdução}\label{sec:introducao}
Alguns links interessantes para se trabalhar com a classe abn\TeX\ e \LaTeX\ em geral\footnote{E também para usar alguns comandos de citação como exemplo}:
\begin{alineas}
  \item Informações da classe Abn\TeX : \citeonline{abntex2classe}
  \item Ajustes nas citações e referências: \citeonline{abntex2cite} e \citeonline{abntex2cite-alf}
  \item Classe memoir (base do Abn\TeX\ ): \apudonline{memoir}{abntex2classe}
  \item Livros interessantes sobre \LaTeX: \cite{Dongen2012,LeslieLamport90,FrankMittelbach111,Dongen2012}
  \item Distribuição \LaTeX\ para windows: \url{http://miktex.org/}
  \item Editor \LaTeX\ gratuito: \url{http://texstudio.sourceforge.net/}
  \item Gerenciador de arquivos \texttt{.bib}: \url{http://jabref.sourceforge.net/}
  \item Gerenciador de artigos: \url{http://www.mendeley.com/}
  \item Exemplo de Tabela: Veja \autoref{tab:cronograma}
\end{alineas}

\section{Objetivo Geral}

\section{Objetivos Específicos}

\chapter{Referencial Teórico}\label{sec:RefTeorico}
\begin{figure}[htbp]
  \begin{center}
  \includegraphics[width=.5\linewidth]{logoufv}\\
  \end{center}
  \caption[Exemplo de Figura]{Exemplo de inserção de figura no \LaTeX. A legenda deve vir abaixo da figura. Pode usar o comando \texttt{\textbackslash legend} ou \texttt{\textbackslash fonte} para inserir a fonte da figura. Observe que na lista de ilustrações foi utilizado o nome curto fornecido como parâmetro do caption da figura (veja o arquivo fonte .tex) ao invés dessa legenda estupidamente extensa feita de forma proposital}
  \label{fig:logo}
  \legend{Fonte: Próprio Autor}
\end{figure}

\chapter{Trabalhos Relacionados}\label{sec:TrabRel}

\chapter{Medodologia e Cronograma}\label{sec:metodos}

O abn\TeX\ introduziu o comando \texttt{IBGEtab} para formatação de tabelas. Um exemplo de tabela convencional do \LaTeX\ pode ser observado na \autoref{tab:cronograma} enquanto um exemplo usando o \texttt{IBGEtab} é mostrado na Tabela \ref{tab:cronogramaIBGE}.

\begin{table}[htbp]
  \centering
    \caption[Cronograma Normal]{Cronograma do Projeto em Meses}
    \label{tab:cronograma}
    \begin{tabular}{lcccccccccccc} %|c|c|c|c|c|c|c|c|c|c|c|c
    \toprule
    \textbf{Atividade} & \textbf{1} & \textbf{2} & \textbf{3} & \textbf{4} & \textbf{5} & \textbf{6} & \textbf{7} & \textbf{8} & \textbf{9} & \textbf{10} & \textbf{11} & \textbf{12} \\
    \midrule
        Revisão Bibliográfica & $\bullet$ & $\bullet$ & & & & & & & & & & \\
        Métodos & & & $\bullet$ & $\bullet$ & & & & & & & & \\
        Testes & & & & $\bullet$ & $\bullet$ & $\bullet$ & & & & & & \\
        Resultados & & & & & & & $\bullet$ & $\bullet$ & & & & \\
        Conclusão & & & & & & & $\bullet$ & $\bullet$ & $\bullet$ & & & \\
        Banca & & & & & & &&&& $\bullet$ & $\bullet$ & $\bullet$ \\
    \bottomrule
    \end{tabular}%
    \fonte{Próprio Autor}
\end{table}%

\begin{table}[htbp]
    \IBGEtab{
    \caption[Cronograma (IBGE)]{Cronograma do Projeto em Meses usando o comando IBGEtab para a formatação da tabela}
    \label{tab:cronogramaIBGE}
    }{
    \begin{tabular}{lcccccccccccc} %|c|c|c|c|c|c|c|c|c|c|c|c
    \toprule
    \textbf{Atividade} & \textbf{1} & \textbf{2} & \textbf{3} & \textbf{4} & \textbf{5} & \textbf{6} & \textbf{7} & \textbf{8} & \textbf{9} & \textbf{10} & \textbf{11} & \textbf{12} \\
    \midrule
        Revisão Bibliográfica & $\bullet$ & $\bullet$ & & & & & & & & & & \\
        Métodos & & & $\bullet$ & $\bullet$ & & & & & & & & \\
        Testes & & & & $\bullet$ & $\bullet$ & $\bullet$ & & & & & & \\
        Resultados & & & & & & & $\bullet$ & $\bullet$ & & & & \\
        Conclusão & & & & & & & $\bullet$ & $\bullet$ & $\bullet$ & & & \\
        Banca & & & & & & &&&& $\bullet$ & $\bullet$ & $\bullet$ \\
    \bottomrule
    \end{tabular}%
    }{
    \fonte{Próprio Autor}}
\end{table}%

\chapter{Resultados Esperados}\label{sec:resultEsperados}

% ---


% ----------------------------------------------------------
% ELEMENTOS PÓS-TEXTUAIS
% ----------------------------------------------------------

\postextual

% Referências Bibliográficas
\bibliography{referencias}

% Caso sejam necessários apêndices ou anexos em seu documento, use os ambientes abaixo

%% Apêndices
\comment{ % Remova essa linha para descomentar
\begin{apendicesenv}
  \partapendices

  \chapter{Primeiro Apêndice}
  \lipsum[30]
  \chapter{Segundo Apêndice}
  \lipsum[30]
\end{apendicesenv}
} % Remova essa linha para descomentar

%% Anexos
%\begin{anexosenv}
\comment{ % Remova essa linha para descomentar
  \partanexos

  \chapter{Primeiro Anexo}
  \lipsum[30]
  \chapter{Segundo Anexo}
  \lipsum[31]
\end{anexosenv}
} % Remova essa linha para descomentar

\end{document}
